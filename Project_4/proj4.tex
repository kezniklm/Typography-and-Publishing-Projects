%ITY - Projekt 4
%Autor: Matej Keznikl
%Fakulta: Fakulta informačných technológií VUT v Brne (FIT VUT)

\documentclass[a4paper, 11pt]{article}
\usepackage[text={17cm, 24cm}, left=2cm, top=3cm]{geometry}
\usepackage[slovak]{babel}
\usepackage[utf8]{inputenc}
\usepackage[hidelinks, unicode]{hyperref}

\begin{document}

\begin{titlepage}
	\begin{center} 
		\textsc{\Huge Vysoké učení technické v~Brně\\
			\huge Fakulta informačních technologií \\}
			\vspace{\stretch{0.382}}
			\LARGE Typografie a publikování\,--\ 4. projekt\\
			\Huge Tabulky a obrázky
			\vspace{\stretch{0.618}}
			\end{center}
			\Large \today \hfill Matej Keznikl
\end{titlepage}


%kniha
\section{Typografia}
Typografia sa zaoberá dizajnom a vytváraním textu s cieľom zlepšiť jeho čitateľnosť, porozumenie a estetickú príjemnosť. To sa dosahuje pomocou rôznych prvkov, ako sú typy písma, veľkosť písma, odsadenie od okraja, riadkovanie, interpunkcia, farby a iné. Typografia sa teda zameriava nielen na estetiku, ale aj na funkčnosť textu a jeho schopnosť efektívne komunikovať. \cite{KNUTH}


%clanok
\subsection{Design webových strániek}
Typografia sa taktiež používa v digitálnych médiách, najmä na webových stránkach a mobilných aplikáciách. Použitie vhodného písma a jeho správne použitie môže mať veľký vplyv na celkový vizuálny dojem a používateľskú skúsenosť s webovou stránkou alebo mobilnou aplikáciou. Je dôležité prispôsobiť typografiu cieľovej skupine a účelu projektu a riešiť zobrazenie textu na rôznych zariadeniach. Článok môže byť užitočný pre dizajnérov a vývojárov webových stránok a mobilných aplikácií, ktorí chcú zlepšiť funkčnosť a používateľskú skúsenosť svojich digitálnych projektov. \cite{WebsiteDesign}


%praca
\subsection{Prezentácia}
Typografia môže byť účinným nástrojom na zlepšenie prezentácie informácií. Výber vhodnej typografie pre konkrétny formát a použitie rôznych typografických prvkov, ako sú veľkosť písma, farby a medzery, môžu výrazne zlepšiť čitateľnosť a atraktívnosť textu. Použitie týchto prvkov môže zvýšiť efektivitu prezentovania informácií a urobiť tak prezentáciu pre čitateľa príjemnejšou a prehľadnejšou. Tieto princípy môžu byť aplikované v rôznych médiách a môžu byť užitočné pre všetkých, ktorí sa zaoberajú tvorbou a prezentovaním informácií, ako sú novinári, grafici a marketéri. \cite{NovotnaMaria2019}

%kniha - monografia
\subsection{Pocity a vnímanie}
Lupton tiež hovorí o vplyve typografie na naše vnímanie a emócie. Napríklad, ak sa písmo použije v príliš malom alebo veľkom rozmeroch alebo ak je písmo ťažko čitateľné, môže to mať negatívny vplyv na to, ako čitateľ vníma správu. V opačnom prípade, ak je typografia dobre navrhnutá, môže to pomôcť zdôrazniť význam správy a urobiť ju prístupnejšou pre čitateľov.

Celkovo, Lupton zdôrazňuje, že typografia je kľúčovým prvkom dizajnu a umenia a jej vhodné použitie môže pomôcť posilniť vizuálne posolstvo a úspešne komunikovať s publikom. \cite{LUPTON}







\newpage
\bibliographystyle{czechiso}
\renewcommand{\refname}{Literatúra}
\bibliography{proj4}
\end{document}