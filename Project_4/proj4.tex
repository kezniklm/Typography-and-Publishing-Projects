%ITY - Projekt 4
%Autor: Matej Keznikl
%Fakulta: Fakulta informačných technológií VUT v Brne (FIT VUT)

\documentclass[a4paper, 11pt]{article}
\usepackage[text={17cm, 24cm}, left=2cm, top=3cm]{geometry}
\usepackage[slovak]{babel}
\usepackage[utf8]{inputenc}
\usepackage[hidelinks, unicode]{hyperref}

\begin{document}

\begin{titlepage}
	\begin{center} 
		\textsc{\Huge Vysoké učení technické v~Brně\\
			\huge Fakulta informačních technologií \\}
			\vspace{\stretch{0.382}}
			\LARGE Typografie a publikování\,--\ 4. projekt\\
			\Huge Tabulky a obrázky
			\vspace{\stretch{0.618}}
			\end{center}
			\Large \today \hfill Matej Keznikl
\end{titlepage}



\section{Typografie}
Typografia sa zaoberá dizajnom a vytváraním textu s cieľom zlepšiť jeho čitateľnosť, porozumenie a estetickú príjemnosť. To sa dosahuje pomocou rôznych prvkov, ako sú typy písma, veľkosť písma, odsadenie od okraja, riadkovanie, interpunkcia, farby a iné. Typografia sa teda zameriava nielen na estetiku, ale aj na funkčnosť textu a jeho schopnosť efektívne komunikovať. \cite{KNUTH}

Typografia sa používa v digitálnych médiách, najmä na webových stránkach a mobilných aplikáciách. Použitie vhodného písma a jeho správne použitie môže mať veľký vplyv na celkový vizuálny dojem a používateľskú skúsenosť s webovou stránkou alebo mobilnou aplikáciou. Je dôležité prispôsobiť typografiu cieľovej skupine a účelu projektu a riešiť zobrazenie textu na rôznych zariadeniach. Článok môže byť užitočný pre dizajnérov a vývojárov webových stránok a mobilných aplikácií, ktorí chcú zlepšiť funkčnosť a používateľskú skúsenosť svojich digitálnych projektov. \cite{WebsiteDesign}

\end{document}