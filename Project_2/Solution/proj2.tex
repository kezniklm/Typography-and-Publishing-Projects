%ITY - Projekt 2 
%Autor: Matej Keznikl
%Fakulta: Fakulta informačných technológií VUT v Brne (FIT VUT)

\documentclass[a4paper, twocolumn, 11pt]{article}
\usepackage[text={18.2cm, 25.2cm}, left=1.4cm, top=2.3cm]{geometry}
\usepackage[czech]{babel}
\usepackage[utf8]{inputenc}
\usepackage[IL2]{fontenc}
\usepackage[unicode,hidelinks]{hyperref}
\usepackage{times}
\usepackage{amsmath}
\usepackage{amsthm}
\usepackage{amssymb}


\begin{document}
\begin{titlepage}
    \begin{center}
        {\Huge\textsc{Vysoké učení technické v Brně \\[0.5em]}}
        {\huge\textsc{Fakulta informačních technologií}}\\
        \vspace{\stretch{0.382}}
        {\LARGE Typografie a publikování\,--\,2. projekt \\[0.4em]
        Sazba dokumentů a matematických výrazů}\\
        \vspace{\stretch{0.618}}
        {\Large 2023 \hfill Matej Keznikl (xkezni01)}
    \end{center}
\end{titlepage}

\newtheorem{definicia}{Definice}
\newtheorem{veta}{Věta}

\section*{Úvod}

V~této úloze si vyzkoušíme sazbu titulní strany, matematických vzorců, prostředí a~dalších textových struktur obvyklých pro technicky zaměřené texty\,--\,například Definice~\ref{definicia} nebo rovnice~\eqref{rovnica3} na straně~\pageref{definicia}. Pro vytvoření těchto odkazů používáme kombinace příkazů \verb|\label|, \verb|\ref|, \verb|\eqref| a~\verb|\pageref|. Před odkazy patří nezlomitelná mezera. Pro zvýrazňování textu jsou zde několikrát použity příkazy \verb|\verb| a \verb|\emph|. 

Na titulní straně je použito prostředí \texttt{titlepage} a~sázení nadpisu podle optického středu s~využitím \emph{přesného} zlatého řezu. Tento postup byl probírán na přednášce. Dále jsou na titulní straně použity čtyři různé velikosti písma a~mezi dvojicemi řádků textu je použito odřádkování se zadanou relativní velikostí 0,5 em a 0,4 em\footnote[1]{Nezapomeňte použít správný typ mezery mezi číslem a~jednotkou.}.

\section{Matematický text}
V~této sekci se podíváme na sázení matematických symbolů a~výrazů v~plynulém textu pomocí prostředí \texttt{math}. Definice a~věty sázíme pomocí příkazu \verb|\newtheorem| s~využitím balíku \texttt{amsthm}. Někdy je vhodné použít konstrukci \verb|${}$| nebo \verb|\mbox{}|, která říká, že (matematický) text nemá být zalomen. 

\label{definicia}
\begin{definicia}
\emph{Zásobníkový automat} (ZA) je definován jako sedmice tvaru $A =(Q,\Sigma,\Gamma,\delta,q_0,Z_0,F)$, kde: 
\begin{itemize}
        \item $Q$ je konečná množina \emph{vnitřních (řídicích) stavů}, 
        \item $\Sigma$ je konečná \emph{vstupní abeceda}, 
        \item $\Gamma$ je konečná \emph{zásobníková abeceda}, 
        \item $\delta$ je \emph{přechodová funkce} $Q \times (\Sigma \cup \{\epsilon\}) \times \Gamma \rightarrow 2^{Q \times \Gamma^*}$,
        \item $q_0 \in Q$ je \emph{počáteční stav}, $Z_0 \in \Gamma$ je \emph{startovací symbol zásobníku }a $F \subseteq Q$ je množina \emph{koncových stavů.} 
\end{itemize}
\end{definicia}

Nechť $P =(Q,\Sigma,\Gamma,\delta,q_0,Z_0,F)$ je ZA. \emph{Konfigurací} nazveme trojici $(q, w, \alpha) \in Q \times \Sigma^* \times \Gamma^*$, kde $q$ je aktuální stav vnitřního řízení, $w$ je dosud nezpracovaná část vstupního řetězce a~$\alpha = Z_{i_1}Z_{i_2} \dots Z_{i_k}$ je obsah zásobníku.

\subsection{Podsekce obsahující definici a~větu}
\begin{definicia}
\emph{Řetězec} $w$~\emph{nad abecedou} $\Sigma$~\emph{je přijat ZA} A~jestliže $(q_0, w, Z_0) \overset{\ast}{\underset{A}{\vdash}} (q_F,\epsilon, \gamma)$ pro nějaké $\gamma \in \Gamma^*$ a $q_F \in F$. Množina $L(A) = \{w\ |\ w $ je přijat ZA A\} $\subseteq \Sigma^*$ je \emph{jazyk přijímaný ZA} A.
\end{definicia}
\begin{veta}
Třída jazyků, které jsou přijímány ZA, odpovídá \emph{bezkontextovým jazykům}.
\end{veta}

\section{Rovnice}
Složitější matematické formulace sázíme mimo plynulý text pomocí prostředí \texttt{displaymath}. Lze umístit i~několik výrazů na jeden řádek, ale pak je třeba tyto vhodně oddělit, například příkazem \verb|\quad|. 
$$
1^{2^3} \neq \Delta_{\Delta_{\Delta^3}^2}^1 \quad y_{22}^{11}-\sqrt[9]{x+\sqrt[7]{y}} \quad x>y_1\leq y^2 
$$
V rovnici \eqref{rovnica2} jsou využity tři typy závorek s různou \emph{explicitně} definovanou velikostí. Také nepřehlédněte, že nasledující tři rovnice mají zarovnaná rovnítka, a~použijte k~tomuto účelu vhodné prostředí. 
\begin{eqnarray}
-cos^2\beta &=& \frac{\frac{\frac{1}{x}+\frac{1}{3}}{y}+1000}{\prod\limits_{j=2}^{8}q_j}\label{rovnica1}\\
\bigg(\Big \{ b \star \bigl [ 3 \div 4 \bigl]\circ \ a \Big \}^\frac{2}{3} \bigg) \quad &=& \log_{10} x \label{rovnica2}\\
\int_{a}^b f(x)\ dx &=& \int_{c}^d f(y)\ dy \label{rovnica3}
\end{eqnarray}
V~této větě vidíme, jak vypadá implicitní vysázení limity $\lim_{m\rightarrow\infty}f(m)$ v~normálním odstavci textu. Podobně je to i~s~dalšími symboly jako $\bigcup_{N\in \mathcal{M}} N$ či $\sum_{i=1}^m x_{i}^2$. S vynucením méně úsporné sazby příkazem \verb|\limits| budou vzorce vysázeny v podobě $\lim\limits_{m\rightarrow\infty}f(m)$ a $\sum\limits_{i=1}^m x_{i}^4$.

\section{Matice}
Pro sázení matic se velmi často používá prostředí \texttt{array} a~závorky (\verb|\left|, \verb|\right|). 
$$
\mathbf{B}=\left|\begin{array}{cccc}
b_{11} & b_{12} & \cdots & b_{1 n} \\
b_{21} & b_{22} & \cdots & b_{2 n} \\
\vdots & \vdots & \ddots & \vdots \\
b_{m 1} & b_{m 2} & \cdots & b_{m n}
\end{array}\right|=\left|\begin{array}{cc}
t & u \\
v & w
\end{array}\right|=t w-u v
$$
$$
\mathbb X = \mathbf {Y} \Longleftrightarrow \bigg [ \begin{array}{ccc}
     & \Omega +\Delta & \hat{\psi}  \\
  \vec{\pi}   &  \omega
\end{array}\bigg ] \neq 42
$$
Prostředí \texttt{array} lze úspěšně využít i~jinde, například na pravé straně následující rovnice. Kombinační číslo na levé straně vysázejte pomocí příkazu \verb|\binom|.
$$
\binom{n}{k}=\left\{\begin{array}{cl}
0 & \text {pro } k < 0 \\
\frac{n !}{k !(n-k) !} & \text {pro } 0 \leq k \leq n \\
0 & \text {pro } k>0 
\end{array}\right.
$$


\end{document}